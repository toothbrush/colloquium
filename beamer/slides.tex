\documentclass[]{beamer}
\usetheme{Warsaw}
% kill another warning:
\usepackage{textcomp}
\usepackage{verbatim}
\usepackage{graphicx}
\usepackage{latexsym}
\usepackage{amsfonts}
\usepackage{tikz}
\usepackage{dsfont}
\usepackage{listings}

% get rid of LaTeX-beamer warning:
\let\Tiny=\tiny
% hm a bit ugly but ok:
\usepackage[T1]{fontenc}

\author{Paul van der Walt}
\institute{Utrecht University}
\date{15th December 2012}
\title[Binders Unbound]{Binders Unbound\\\small{by Weirich, Yorgey and Sheard}}
\begin{document}

\begin{frame}
    \maketitle
\end{frame}

%TODO: question-> talk about process of refining eg let* or just give final definition?

%% OUTLINE:
% intro / problem statement
% solution: unbound DSL
% introducing unbound
% feature: binding patterns
% semantics
% theorems
% implementation

% discussion

% related work
% (discussion?)
% conclusion / summary / contribution

\begin{frame}
    \tableofcontents
\end{frame}

\begin{frame}{Problem}
    \begin{itemize}
        \item $\alpha$-equivalence and capture-avoiding substitution are conceptually simple
        \item Implementation is a pain
        \item Subtle, boring \emph{boilerplate} code, prone to errors
        \item Maintenance hassle
    \end{itemize}
\end{frame}

\begin{frame}{Solution}
    \begin{itemize}
        \item Enter Unbound
        \item Out-of-the-box support for these problematic operations
    \end{itemize}
\end{frame}

\begin{frame}{Concrete example}
    \begin{example}
        \begin{spec}
            test = undefined
        \end{spec}
    \end{example}
\end{frame}



\begin{frame}
    hello\cite{weirich2011binders}
\end{frame}


\begin{frame}
    \bibliography{bu}{} %2nd parm is width
    \bibliographystyle{plain}
    
\end{frame}
\end{document}
